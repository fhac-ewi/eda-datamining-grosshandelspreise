\vorlesung{Thema der ersten Vorlesung}{31.01.2021}
Inhaltlich ist in der ersten Vorlesung nichts passiert. I'm sorry!

\vorlesung{Beispiele in LaTeX}{06.02.2021}

So funktioniert eine Aufzählung in LaTeX:

\begin{enumerate}
  \item Das ist ein Punkt.
  \item Das ist ein zweiter Punkt.
\end{enumerate}


So funktioniert eine stichpunktartige Auflistung in LaTeX:

\begin{itemize}
  \item Das ist ein Stichpunkt.
  \item Das ist ein zweiter Stichpunkt.
\end{itemize}

So könnte eine Begriffserklärung in LaTeX aussehen:

\begin{description}
  \item \textbf{Stichpunkt} Kurzer präziser Punkt.
  \item \textbf{Punkt} Ein wundervolles Zeichen. Findet Verwendung auf dem i und am Ende eines Satzes.
\end{description}

Um ein Bild einzufügen, muss dieses zuerst in dem Ordner \textit{images} abgelegt werden.
Anschließend können diese mit dem \textit{image} Befehl eingebunden werden.

\image{fh-logo-right.pdf}{Logo der FH Aachen}

Dies ist ein weiterer Absatz.
