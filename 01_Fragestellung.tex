\section{Wissenschaftliche Fragestellung}
Test 

\todo{Kurzes Intro zum Day-Ahead-Markt und energiewirtschaftlicher Relevanz des Untersuchungsobjekts}


\subsection{Zielsetzung und Herangehensweise}
Ziel dieser Arbeit ist die Auswahl eines finalen Modells zur Vorhersage von Großhandelspreisen am deutschen Strommarkt auf Basis der bis dahin bekannten Untersuchungsobekte (\todo{Link zu Untersuchungsobjekte}).

Dazu werden zuerst die Untersuchungsobjekte analysiert und die Relevanz für den deutschen Großhandelspreis verdeutlicht. 

Anschließend werden unter Verwendung der relevanten Untersuchungsobjekte mehrere Modelle erstellt, welche den Großhandelspreis vorhersagen sollen. Zuerst werden dafür Regressionsmodelle erstellt und hinsichtlich ihrer Genauigkeit miteinander verglichen. Das genauste Regressionsmodell wird als Benchmark für die deutlich komplexeren Modelle des vertieften Lernens (Deep-Learning) verwendet, welche im zweiten Schritt erstellt werden. Abschließend wird das beste Modell aus allen erstellten Modellen ausgewählt.


\todo{Konkretisierung der Aufgabenstellung und Gliederung der Arbeit (Herangehensweise )}
