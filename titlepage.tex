

\begin{titlepage}
  
  \thispagestyle{empty}
  \newgeometry{right=0cm,left=0cm, top=0cm, bottom=0cm}
  
  \makebox[0pt][l]{
    \begin{picture}(50,50)
      % 100 vergrößern, damit logo hoch wandert 
      \put(543, -155){
        \hfill
        \includegraphics[width=1.7cm]{./images/fh-logo-right.pdf} %fhac-right.pdf} %FHAC.jpg} 
      }
    \end{picture}
  }

  \begin{center}
    {\Large \textsc{\textbf{FH Aachen}}}\\[0.2cm]
    {\large \textsc{University of Applied Sciences}}\\
    {\large \textsc{Energiewirtschaft \& Informatik (M. Sc.)}}\\[1cm]

    {\Huge \textsc{\textbf{Vorhersage von Großhandelspreisen}}}\\[0.2cm]
    {\Large \textsc{Seminararbeit SS2021}}\\[0.2cm]
    {\Large \textsc{Energiedatenanalyse - Datamining}}\\[0.2cm]
    {\textsc{Paul Krüger, Vivekanantha Kumar}}\\
  \end{center}


  \vspace{2cm}
  

  \noindent\makebox[\textwidth][c]{
    \begin{minipage}{0.6\textwidth}
      \rule{\textwidth}{0.5pt}\\[.3cm]
      \section*{Vorwort}
      Begleitend zu dieser Seminararbeit gibt es ein Matlab Skript, welches alle der hier vorgestellten Analysen und Modellierungen enthält. Die verwendeten Rohdaten sowie Cache-Files lieben ebenfalls dieser Arbeit bei.

      Das Skript wurde mit MatLab 2020b erstellt. Zusätzlich werden für die Ausführung des Skripts folgende Pakete benötigt:
      \begin{itemize}
        \item Statistics and Machine Learning Toolbox v12.1
        \item Mapping Toolbox v5.0
        \item Parallel Computing Toolbox v7.3
        \item Deep Learning Toolbox v14.1
        \item Optimization Toolbox v9.1
        \item Reinforcement Learning Toolbox v2.0
      \end{itemize}
      Während unseres Tests konnten nicht alle Code-Sektionen auf einmal ausgeführt werden. MatLab hat sich dabei leider an der exists-Funktion aufgehangen. Die einzelnen Code Sektionen können jedoch problemlos einzeln nacheinander ausgeführt werden.
  \end{minipage}
  }

  \end{titlepage}
 
  \newpage
  \tableofcontents