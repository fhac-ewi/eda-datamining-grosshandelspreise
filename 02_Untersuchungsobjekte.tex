\section{Auswahl Untersuchungsobjekte}
-	Ableitung möglicher relevanter Objekte aus technoökonomischen Zusammenhang (Struktur des Energiemarkts, Aufbau Merit Order)
-	Auswahl der Datenquellen und deren Zeitreihen
-	Strukturierung der Daten (was wird in welcher Form dokumentiert (Granularität/Auflösung, Fristigkeit))
-	Ableitung der relevanten Zeitreihen
Datenquelle: \url{https://www.smard.de/home/downloadcenter/download-marktdaten#!?downloadAttributes=%7B%22selectedCategory%22:2,%22selectedSubCategory%22:5,%22selectedRegion%22:%22DE-LU%22,%22from%22:1618869600000,%22to%22:1619819999999,%22selectedFileType%22:%22XLS%22%7D}


Für die Auswahl der Untersuchungsobjekte wurden Daten von SMARD heruntergeladen. Die Seite stellt viele Zeitreihen zur Verfügung.
Die folgende Abbildung soll verdeutlichen, zu welchem Zeitpunkt die Daten vorliegen und inwieweit diese für unsere Analyse verwendet werden können.

\image{Auswahl_Rohdaten.png}{Zeitreihen und deren Verfügbarkeit von SMARD}

- Zeitraum 2018 - 2020 heruntergeladen
- Festegestellt, dass 2018 große Datenlücken aufweist (vergl. Matlab)
  - Zeitreihen beginnen erst ab 1.10.2018
  - Zeitreihen sind Lückenhaft
  
