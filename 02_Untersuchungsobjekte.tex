\section{Auswahl Untersuchungsobjekte}
In diesem Abschnitt werden mögliche relevante Datenreihen unter Berücksichtigung technoökonomischer Zusammenhänge beschafft und strukturiert.

\subsection{Beschaffung der Rohdaten}
Für die Auswahl der Untersuchungsobjekte wurden diverse Daten von der Webseite SMARD\footnote{Strommarktdaten, Stromhandel und Stromerzeugung in Deutschland: \url{https://smard.de}} heruntergeladen. Diese Webseite wird von der Bundesnetzagentur betrieben und stellt verschiedenste Strommarktdaten gemäß § 111d EnWG\footnote{§ 111d EnWG: \url{https://gesetze-im-internet.de/enwg_2005/__111d.html}} der Öffentlichkeit zur kostenfreien Nutzung zur Verfügung.

\begin{table}[h]
  \centering
  \begin{tabularx}{\textwidth}{|l|X|c|l|}
    \hline

    \textbf{Bezeichnung} & \textbf{Beschreibung} & \textbf{Einheit} & \textbf{Auflösung} \\ \hline
    
    Großhandelspreise & Preise von verschiedenen Strommarktbörsen innerhalb Europas. & € & Stündlich \\ \hline

    Außenhandel & Gehandelte importierte und exportierte Energiemengen gegenüber angrenzenden Strommärkten. & MWh & Stündlich \\ \hline

    Stromfluss & Physikalischerer Stromfluss importierter und exportierter Energiemengen gegenüber angrenzenden Strommärkten. & MWh & Stündlich \\ \hline

    Prognose Erzeugung & Prognose für Stromerzeugungen für verschiedene Anlagetypen. & MWh & Viertelstündlich \\ \hline

    Prognose Verbrauch & Prognose für den gesamten Stromverbrauch und die Residuallast. & MWh & Viertelstündlich \\ \hline

    Erzeugung & Realisierte Erzeugung je Anlagentyp. & MWh & Viertelstündlich \\ \hline

    Verbrauch & Realisierter Stromverbrauch und Residuallast. & MWh & Viertelstündlich \\ \hline

    Brennstoffkosten & Preise von Gas, Kohle und CO2 für Deutschland. & € & Täglich \\ \hline
  \end{tabularx}
  \caption{Verfügbare Datenreihen}
\end{table}

Aus technoökonomischen Aspekten haben all diese Daten einen Einfluss auf den Großhandelspreis, da diese Zeitreihen Angebot und Nachfrage auf dem deutschen Markt wiederspiegeln. Auf dessen Basis wird mithilfe der Merit Order der deutsche Großhandelspreis gebildet. Die Brennstoffkosten\footnote{Brennstoffkosten aus Ilias übernommen} von Gas, Kohle und CO2 beschreiben Brennstoffkosten und stellen somit teilweise die Erzeugungskosten des Angebots dar.

Als Untersuchungszeitraum wurde der 1. Januar 2018 bis 31. Dezember 2020 festgelegt. Alle Daten wurden, sofern diese verfügbar waren, für diesen Zeitraum heruntergeladen.

Für die Vorhersage des Großhandelspreises können nur Daten verwendet werden, welche zum Zeitpunkt der Vorhersage bereits vorliegen. Es kann beispielsweise nicht die Menge der realisierten Erzeugung zum Liefertermin 31. Januar 2019 12:00 Uhr für die Vorhersage des Großhandelspreises zu diesem Liefertermin verwendet werden, da der Handel vor dem Liefertermin erfolgt und die realisierte Erzeugung erst nach dem Liefertermin bekannt ist. Die folgende Abbildung soll verdeutlichen, zu welchem Zeitpunkt welche Datenreihen vorliegen und inwieweit diese für unsere Analyse verwendet werden können.

\image{Auswahl_Rohdaten.png}{Zeitreihen und deren Verfügbarkeit von SMARD}

Da Brennstoffkosten für Gas, Kohle und CO2 jeweils am Vortag gehandelt werden, stehen diese Analog wie die Prognosen vor dem Handelszeitpunkt zur Verfügung.

\subsection{Datenbereinigung}
Bei der ersten Sichtung der Rohdaten ist aufgefallen, dass einige Datenreihen lücken- oder fehlerhaft sind. Dies muss bereinigt werden, damit mit den Daten weitergearbeitet werden kann. Zudem müssen alle Daten auf die gleiche Auflösung (Viertelstunden-, Stunden- oder Tageswerte) gebracht werden.

\subsubsection*{Umgang mit Datenlücken}
Einige Datenreihen, wie zum Beispiel die Prognosen, werden erst seit dem 1. Oktober 2018 erhoben und fehlen deshalb vor diesem Datum. Auch andere Datenreihen weisen im Jahr 2018 vereinzelte Datenlücken auf, welche mit Ersatzwerten aufgefüllt werden müssten. Ersatzwerte verfälschen das Ergebnis der Analyse, da diese lediglich eine Schätzung sind. Aus diesen Gründen wurde der Untersuchungszeitraum auf den 1. Januar 2019 bis 31. Dezember 2020 angepasst.

Bei dem prognostizierten Stromverbrauch fehlten einzelne Werte im Untersuchungszeitraum. Diese Lücken wurden durch den Wert des Folgetages (+24 Stunden) aufgefüllt. 

\subsubsection*{Stündliche Auflösung}
Die Daten für Prognose und Realisierung der erzeugten und verbrauchten Energiemengen liegen in viertelstündlicher Auflösung vor. Diese Zeitreihen werden zu stündlichen Werten aggregiert. Eine besondere Herausforderung war dabei der Umgang mit der Zeitumstellung. Die von der Webseite SMARD bereitgestellten Zeitangaben sind im deutschen Format, weshalb es doppelte Zeitstempel gibt. Die Daten wurden entsprechend rekonstruiert.

Die Marktpreise für Gas, Kohle und CO2 liegen als Tageswerte vor. Die Preise dieser drei Güter wurden auf alle 24 Stunden des Tages (bzw. mal 23 oder 25 Stunden aufgrund der Zeitumstellung) dupliziert, da diese für den gesamten Liefertag gelten. 

\subsubsection*{Wegfall unplausibler Datenreihen}
Einige Zeitreihen enthalten viele fehlende oder unplausible Werte. Da diese Lücken nicht rekonstruiert werden können, werden die Zeitreihen nicht verwendet. Dies betrifft beispielsweise den Großhandelspreis von Dänemark, Tschechien und Ungarn. 

\subsubsection*{Zeitumstellung}
Die Zeitumstellung bleibt in den Zeitreihen enthalten. Dies bedeutet, dass im Frühjahr ein Tag mit nur 23 Stunden existiert und im Herbst ein Tag mit 25 Stunden. Dies wird die Daten bis 7 Tage nach der Zeitumstellung beeinflussen. (Die Leute stehen weiterhin zur gleichen Uhrzeit auf (bzw. stellen sich auf die neue Uhrzeit ein / die Prognosen hingegen von erneuerbaren Energien stellen sich nicht schlagartig um - das ist ein schleichender Prozess übers gesamte Jahr) Aus diesem Grund erfolgt keine Bereinigung/Ersatzwertbildung. Damit würde man meiner Meinung nach das Problem nur schlimmer machen.

\subsection{Datenauswahl}
Nach der visuellen Analyse und der Bereinigung der Rohdaten wird daraus eine Datenbasis für die weiteren Analysen und Modellierung geschaffen. Aufgrund der Lücken wird nur der Zeitraum 1. Januar 2019 bis 31. Dezember 2020 verwendet.

Unter Berücksichtigung der Verfügbarkeit (vgl. Abschnitt Beschaffung der Rohdaten) werden die Daten jeweils dem zu prognostizierenden Großhandelspreis gegenübergestellt. 

Die nachfolgende Analysen werden losgelöst vom Liefertermin (im Bezug auf das Datum) erfolgen, da das Modell die Zusammenhänge und Trends auf Basis der Features und nicht anhand des Liefertags erlernen soll. Ist dies nicht möglich, ist das ein Anzeichen dafür, dass die gewählten Features den Trend im Großhandelspreis nicht vollständig beschreiben können. Aus energiewirtschaftlicher Sicht gibt es keine Begründung, wieso der Strom am 2. Februar teurer als am 2. August sein sollte. Aus energiewirtschaftlicher Sicht sollte dies durch weitere Features wie beispielsweise Außentemperatur, Sonnenauf-/-untergang, Sonnenstunden und Einstrahlungswinkel beschrieben werden. Die Daten werden deshalb nachfolgend losgelöst der Liefertermins als Daten für Handelszeitschritt t betrachtet. (Uns ist bewusst, dass deutlich bessere Ergebnisse unter hinzunahme des Liefermonats, Liefertags und Lieferstunde erzielt werden könnten.)

Nochmal kurz: Das Modell soll Trends nicht anhand des Datums festmachen, sondern den ``wahren'' Features. Dadurch schreibt das Modell einen eventuell vorhandenen Trend nicht im Zeitlichen Verlauf fort, sondern stützt dies auf die Entwicklung der Marktsituation. 

\image{Einfluss_Zeit.PNG}{Einfluss des Handelszeitpunkts auf den Großhandelspreis}

Der Liefertermin hat natürlich einen Einfluss auf den Großhandelspreis. So werden am Großhandelsmarkt die Produkte Peak (8-20 Uhr) und Offpeak (sonst) gehandelt. Die Preise dieser beiden Produkte weisen eine unterschiedliche Verteilung auf. 
Genauso lassen sich auch unterschiedliche Verteilungen der Preise an Werktagen und Wochenenden feststellen. 

Um dies korrekt in den Modellen abzubilden, müssten separate Modelle erstellt und hinterher zusammengefügt werden. Dies übersteigt meiner Meinung nach aber den Aufwand einer Zweiergruppe, weshalb dies nicht umgesetzt wurde. Jedoch werden diese vier gefundenen Gruppen als Features für die Deep Learning Modelle verwendet. Ob und was die Modelle damit anstellen ist deren Sache.

\todo{Daten in Objekt Basis beschreiben}



-	Ableitung möglicher relevanter Objekte aus technoökonomischen Zusammenhang (Struktur des Energiemarkts, Aufbau Merit Order)
-	Strukturierung der Daten (was wird in welcher Form dokumentiert (Granularität/Auflösung, Fristigkeit))
-	Ableitung der relevanten Zeitreihen
Datenquelle: \url{https://www.smard.de/home/downloadcenter/download-marktdaten#!?downloadAttributes=%7B%22selectedCategory%22:2,%22selectedSubCategory%22:5,%22selectedRegion%22:%22DE-LU%22,%22from%22:1618869600000,%22to%22:1619819999999,%22selectedFileType%22:%22XLS%22%7D}