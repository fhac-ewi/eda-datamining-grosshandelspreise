\section{Auswahl Untersuchungsobjekte}
-	Ableitung möglicher relevanter Objekte aus technoökonomischen Zusammenhang (Struktur des Energiemarkts, Aufbau Merit Order)
-	Auswahl der Datenquellen und deren Zeitreihen
-	Strukturierung der Daten (was wird in welcher Form dokumentiert (Granularität/Auflösung, Fristigkeit))
-	Ableitung der relevanten Zeitreihen
Datenquelle: \url{https://www.smard.de/home/downloadcenter/download-marktdaten#!?downloadAttributes=%7B%22selectedCategory%22:2,%22selectedSubCategory%22:5,%22selectedRegion%22:%22DE-LU%22,%22from%22:1618869600000,%22to%22:1619819999999,%22selectedFileType%22:%22XLS%22%7D}

\subsection{Beschaffung der Rohdaten}
Für die Auswahl der Untersuchungsobjekte wurden diverse Daten von der Webseite SMARD \todo{Link} heruntergeladen. Diese Webseite wird von der Bundesnetzagentur betrieben und stellt verschiedenste Strommarktdaten gemäß § 111d EnWG \todo{https://www.gesetze-im-internet.de/enwg_2005/__111d.html} der Öffentlichkeit zur kostenfreien Nutzung zur Verfügung.

\todo{Auflisten, welche Zeitreihen verwendet wurden und wieso und Auflösung}

Als Untersuchungszeitraum wurde der 1. Januar 2018 bis 31. Dezember 2020 festgelegt. Alle Daten wurden, sofern diese verfügbar waren, für diesen Zeitraum heruntergeladen.

Für die Vorhersage des Großhandelspreises können nur Daten verwendet werden, welche zum Zeitpunkt der Vorhersage bereits vorliegen. Es kann beispielsweise nicht die Menge der realisierten Erzeugung zum Liefertermin 31. Januar 2019 12:00 Uhr für die Vorhersage des Großhandelspreises zu diesem Liefertermin verwendet werden, da der Handel vor dem Liefertermin erfolgt und die realisierte Erzeugung erst nach dem Liefertermin bekannt ist. Die folgende Abbildung soll verdeutlichen, zu welchem Zeitpunkt welche Datenreihen vorliegen und inwieweit diese für unsere Analyse verwendet werden können.

\image{Auswahl_Rohdaten.png}{Zeitreihen und deren Verfügbarkeit von SMARD}

Als weitere Datenquelle wurden die Preise für CO2

\subsection{Datenbereinigung}
Bei der ersten Sichtung der Rohdaten ist aufgefallen, dass einige Datenreihen lücken- oder fehlerhaft sind. Dies muss bereinigt werden, damit mit den Daten weitergearbeitet werden kann. Zudem müssen alle Daten auf die gleiche Auflösung (Vierlstunden-, Stunden- oder Tageswerte) gebracht werden.


Einige Datenreihen, wie zum Beispiel \todo{Beispiel}, werden erst seit dem 1. Oktober 2018 erhoben und fehlen deshalb vor diesem Datum. Auch andere Datenreihen weisen im Jahr 2018 vereinzelte Datenlücken auf, welche mit Ersatzwerten aufgefüllt werden müssten. Ersatzwerte verfälschen das Ergebnis der Analyse, da diese lediglich eine Schätzung sind. Aus diesen Gründen wurde der Untersuchungszeitraum auf den 1. Januar 2019 bis 31. Dezember 2020 angepasst.

TODO
TODO hour of day als Spalte

\subsection{Datenauswahl}


- Zeitraum 2018 - 2020 heruntergeladen
- Festegestellt, dass 2018 große Datenlücken aufweist (vergl. Matlab)
  - Zeitreihen beginnen erst ab 1.10.2018
  - Zeitreihen sind Lückenhaft
- Davon abgesehen Datenpunkte zu Schätzen -> Dies würde das Ergebnis nur verschlechtern.
- Deshalb ZR 1.1.2019 - 31.12.2020 für weitere Analyse verwendet. Hier liegen die Daten halbwegs vollständig vor.

- Probleme mit Zeitumstellung, da hier die lokale Zeit verwendet wird und deshalb Stunden doppelt drinnen sind
  - Vermeintlich einfaches Zusammenfügen von Viertelstunden Daten zu Stundendaten komplex
  - Genauso schwierig von Tageweisen Daten auf Stundendaten zu kommen (andernfalls wäre ein REPEAT 24 easy)

Nachdem der ganze *** sortiert ist, können dann verschiedene Datenreihen extrahiert werden, die für die deskriptive Datenanalyse herangezogen werden können. 
\todo{Datenreihen in Variable Basis beschreiben}
