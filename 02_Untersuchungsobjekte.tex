\section{Auswahl Untersuchungsobjekte}
-	Ableitung möglicher relevanter Objekte aus technoökonomischen Zusammenhang (Struktur des Energiemarkts, Aufbau Merit Order)
-	Auswahl der Datenquellen und deren Zeitreihen
-	Strukturierung der Daten (was wird in welcher Form dokumentiert (Granularität/Auflösung, Fristigkeit))
-	Ableitung der relevanten Zeitreihen
Datenquelle: \url{https://www.smard.de/home/downloadcenter/download-marktdaten#!?downloadAttributes=%7B%22selectedCategory%22:2,%22selectedSubCategory%22:5,%22selectedRegion%22:%22DE-LU%22,%22from%22:1618869600000,%22to%22:1619819999999,%22selectedFileType%22:%22XLS%22%7D}


Für die Auswahl der Untersuchungsobjekte wurden Daten von SMARD heruntergeladen. Die Seite stellt viele Zeitreihen zur Verfügung.
Die folgende Abbildung soll verdeutlichen, zu welchem Zeitpunkt die Daten vorliegen und inwieweit diese für unsere Analyse verwendet werden können.

\image{Auswahl_Rohdaten.png}{Zeitreihen und deren Verfügbarkeit von SMARD}

- Zeitraum 2018 - 2020 heruntergeladen
- Festegestellt, dass 2018 große Datenlücken aufweist (vergl. Matlab)
  - Zeitreihen beginnen erst ab 1.10.2018
  - Zeitreihen sind Lückenhaft
- Davon abgesehen Datenpunkte zu Schätzen -> Dies würde das Ergebnis nur verschlechtern.
- Deshalb ZR 1.1.2019 - 31.12.2020 für weitere Analyse verwendet. Hier liegen die Daten halbwegs vollständig vor.

- Probleme mit Zeitumstellung, da hier die lokale Zeit verwendet wird und deshalb Stunden doppelt drinnen sind
  - Vermeintlich einfaches Zusammenfügen von Viertelstunden Daten zu Stundendaten komplex
  - Genauso schwierig von Tageweisen Daten auf Stundendaten zu kommen (andernfalls wäre ein REPEAT 24 easy)

Nachdem der ganze *** sortiert ist, können dann verschiedene Datenreihen extrahiert werden, die für die deskriptive Datenanalyse herangezogen werden können. 
\todo{Datenreihen in Variable Basis beschreiben}
